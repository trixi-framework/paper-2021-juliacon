
% JuliaCon proceedings template
\documentclass{juliacon}
\setcounter{page}{1}

\begin{document}

\input{header}

\maketitle

\begin{abstract}
TODO: abstract %TODO
\end{abstract}


\section{Introduction}

TODO %TODO
\begin{itemize}
  \item Basic design of Trixi.jl and its capabilities
  \item PID comparison of Fluxo and Trixi.jl
  \item Discussion on Julia for numerical simulation science
  \item Prepare repro repo including scripts for Trixi.jl and Fluxo
  \item Cite \cite{bezanson2017julia,schlottkelakemper2021purely,rackauckas2017differentialequations,revels2016forward}
\end{itemize}



\clearpage
\section*{Some guidelines}
\label{sec:some_guide}

The following notes may help you achieve the best effects with the
\verb juliacon  class file.

\subsection{Writing Julia code}

A special environment is already defined for Julia code,
built on top of \textit{listings} and \textit{jlcode}.

\begin{verbatim}
\begin{lstlisting}[language = Julia]
using Plots

x = -3.0:0.01:3.0
y = rand(length(x))
plot(x, y)
\end{lstlisting}
\end{verbatim}
\begin{lstlisting}[language = Julia]
using Plots

x = -3.0:0.01:3.0
y = rand(length(x))
plot(x, y)
\end{lstlisting}

\subsection{Double Column Figure and Tables}
\label{subsub:double_fig_tab}
For generating the output of figures and tables in double column
we can use the following coding:

\begin{enumerate}
\item For Figures:
\begin{verbatim}
\begin{figure*}...\end{figure*}
\end{verbatim}
\item For landscape figures:
\begin{verbatim}
\begin{sidewaysfigure*}...\end{sidewaysfigure*}
\end{verbatim}
\item For Tables:
\begin{verbatim}
\begin{table*}...\end{table*}
\end{verbatim}
\item For landscape tables:
\begin{verbatim}
\begin{sidewaystable*}...\end{sidewaystable*}
\end{verbatim}
\end{enumerate}

\subsection{Enunciations}
\label{subsub:enunciation}

The \verb juliacon   class file generates the enunciations with the help of
the following commands:
\begin{verbatim}
\begin{theorem}...\end{theorem}
\begin{strategy}...\end{strategy}
\begin{property}...\end{property}
\begin{proposition}...\end{proposition}
\begin{lemma}...\end{lemma}
\begin{example}...\end{example}
\begin{proof}...\end{proof}
\begin{definition}...\end{definition}
\begin{algorithm}...\end{algorithm}
\begin{remark}...\end{remark}
\end{verbatim}
The above-mentioned coding can also include optional arguments
such as
\begin{verbatim}
\begin{theorem}[...]. Example for theorem:
\begin{theorem}[Generalized Poincare Conjecture]
Four score and seven ... created equal.
\end{theorem}
\end{verbatim}

\begin{theorem}[Generalized Poincare Conjecture]
Four score and seven years ago our fathers brought forth,
upon this continent, a new nation, conceived in Liberty,
 and dedicated to the proposition that all men are
created equal.
\end{theorem}


\subsection{Balancing column at last page}
\label{subsub:Balance}
For balancing the both column length at last page use:
\begin{verbatim}
\vadjust{\vfill\pagebreak}
\end{verbatim}

%\vadjust{\vfill\pagebreak}

at appropriate place in your \TeX{} file or in bibliography file.

\subsection{Handling references}
\label{subsub:references}
References are most easily (and correctly) generated using the
BIBTEX, which is easily invoked via
\begin{verbatim}
\bibliographystyle{juliacon}
\bibliography{ref}
\end{verbatim}
When submitting the document source (.tex) file to external
parties, the ref.bib file should be sent with it.
Don't forget to cite \cite{bezanson2017julia}.



% References
\input{bib.tex}

\end{document}

% Inspired by the International Journal of Computer Applications template
